\documentclass[a4paper,11pt]{paper}

\usepackage[utf8]{inputenc}
\usepackage[T1]{fontenc}
\usepackage[margin=3.2cm]{geometry}
\usepackage{enumitem}
\usepackage{CJKutf8}
\usepackage[colorlinks=true,urlcolor=blue,linkcolor=black]{hyperref}
\usepackage{mathtools}
\usepackage{listings}
\usepackage{fancyvrb}
\usepackage{enumitem}
\usepackage{tikz}
\usepackage{listings}
\usepackage{xcolor}
\usepackage{amsmath}
\usepackage{calc}
\usepackage{relsize}
\usepackage{emoji}  % lualatex
\usepackage{fontawesome}  % lualatex
\usepackage{fancyvrb}

\usepackage{lastpage}
\usepackage{fancyhdr}
\pagestyle{fancy}
\fancyhf{} % clear existing header/footer entries
% Place Page X of Y on the right-hand
% side of the footer
\fancyfoot[R]{Page \thepage \hspace{1pt} of \pageref{LastPage}}

\usetikzlibrary{calc,shapes.multipart,chains,arrows}

\renewcommand*{\theenumi}{\thesection.\arabic{enumi}}
\renewcommand*{\theenumii}{\theenumi.\arabic{enumii}}
\let\orighref\href
\renewcommand{\href}[2]{\orighref{#1}{#2\,\smaller[4]\faExternalLink}}

\let\Red=\alert
\definecolor{few-gray-bright}{HTML}{010202}
\definecolor{few-red-bright}{HTML}{EE2E2F}
\definecolor{few-green-bright}{HTML}{008C48}
\definecolor{few-blue-bright}{HTML}{185AA9}
\definecolor{few-orange-bright}{HTML}{F47D23}
\definecolor{few-purple-bright}{HTML}{662C91}
\definecolor{few-brown-bright}{HTML}{A21D21}
\definecolor{few-pink-bright}{HTML}{B43894}

\definecolor{few-gray}{HTML}{737373}
\definecolor{few-red}{HTML}{F15A60}
\definecolor{few-green}{HTML}{7AC36A}
\definecolor{few-blue}{HTML}{5A9BD4}
\definecolor{few-orange}{HTML}{FAA75B}
\definecolor{few-purple}{HTML}{9E67AB}
\definecolor{few-brown}{HTML}{CE7058}
\definecolor{few-pink}{HTML}{D77FB4}

\definecolor{few-gray-light}{HTML}{CCCCCC}
\definecolor{few-red-light}{HTML}{F2AFAD}
\definecolor{few-green-light}{HTML}{D9E4AA}
\definecolor{few-blue-light}{HTML}{B8D2EC}
\definecolor{few-orange-light}{HTML}{F3D1B0}
\definecolor{few-purple-light}{HTML}{D5B2D4}
\definecolor{few-brown-light}{HTML}{DDB9A9}
\definecolor{few-pink-light}{HTML}{EBC0DA}

\colorlet{alert-color}{few-red-bright!80!black}
\colorlet{comment}{few-blue-bright}
\colorlet{string}{few-green-bright}

\lstdefinestyle{ccode}{
    showstringspaces=false,
    stringstyle={\ttfamily\color{string}},
    language=C,escapeinside=`',columns=flexible,commentstyle=\color{comment},
    basicstyle=\ttfamily,
    classoffset=2, keywordstyle=\color{alert-color}
}

\lstnewenvironment{ccode}[1][]%
    {\lstset{style=ccode,basicstyle=\ttfamily\openup-.17\baselineskip,#1}}%
    {}

\lstset{
  basicstyle=\itshape,
  xleftmargin=3em,
  literate={->}{$\rightarrow$}{2}
           {α}{$\alpha$}{1}
           {δ}{$\delta$}{1}
           {ε}{$\epsilon$}{1}
}

\renewcommand{\baselinestretch}{1.1}
\setlength{\parindent}{0pt}
\setlength{\parskip}{1em}

\title{INF333 2023-2024 Spring Semester}
\author{Student 1 Name <email@domain.example>
\\ Student 2 Name <email@domain.example>}

\begin{document}

\maketitle

\section*{\LARGE TP05 Design Document}

The following is the sample design document for this TP. We will provide the general layout and the questions, you will provide answers in the \texttt{quote} environment.


\section{Preliminaries}

If you have any preliminary comments on your submission, notes for the TAs, or extra credit, please give them here.

Please cite any offline or online sources you consulted while preparing your
submission, other than the Pintos documentation, course text, and lecture notes.


\section{Booting Pintos}


\textbf{Q 2.1:} Take screenshots of the successful booting of Pintos in QEMU and Bochs, each in
both the terminal and the QEMU window. Put the screenshots under
\textttt{`pintos/src/p0`}.

\textbf{A 2.1:}

\section{Debugging}

\subsection{Questions About BIOS}


\textbf{Q 2.2:} Your first task in this section is to use GDB to trace the QEMU BIOS a bit to understand how an IA-32 compatible computer boots. Answer the following questions in your design document: 
\begin{itemize}
    \item What is the first instruction that gets executed? 
    \item At which physical address is this instruction located? 
    \item Can you guess why the first instruction is like this?
    \item What are the next three instructions?
\end{itemize}

\textbf{A 2.2:}


\textbf{Q 2.3:}Trace the Pintos bootloader and answer the following questions in your design document:
\begin{itemize}
    \item How does the bootloader read disk sectors? In particular, what BIOS interrupt is used?
    \item How does the bootloader decides whether it finds the Pintos kernel?
    \item What happens when the bootloader could not find the Pintos kernel?
    \item At what point does the bootloader transfer control to the Pintos kernel?
\end{itemize}

\textbf{A 2.3:}


\textbf{Q 2.4:} Add a screenshot of gdb while tracing the Pintos kernel.

\textbf{A 2.4:}






\end{document}
