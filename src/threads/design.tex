\documentclass[a4paper,11pt]{paper}

\usepackage[utf8]{inputenc}
\usepackage[T1]{fontenc}
\usepackage[margin=3.2cm]{geometry}
\usepackage{enumitem}
\usepackage{CJKutf8}
\usepackage[colorlinks=true,urlcolor=blue,linkcolor=black]{hyperref}
\usepackage{mathtools}
\usepackage{listings}
\usepackage{fancyvrb}
\usepackage{enumitem}
\usepackage{tikz}
\usepackage{listings}
\usepackage{xcolor}
\usepackage{amsmath}
\usepackage{calc}
\usepackage{relsize}
\usepackage{emoji}  % lualatex
\usepackage{fontawesome}  % lualatex
\usepackage{fancyvrb}

\usepackage{lastpage}
\usepackage{fancyhdr}
\pagestyle{fancy}
\fancyhf{} % clear existing header/footer entries
% Place Page X of Y on the right-hand
% side of the footer
\fancyfoot[R]{Page \thepage \hspace{1pt} of \pageref{LastPage}}

\usetikzlibrary{calc,shapes.multipart,chains,arrows}

\renewcommand*{\theenumi}{\thesection.\arabic{enumi}}
\renewcommand*{\theenumii}{\theenumi.\arabic{enumii}}
\let\orighref\href
\renewcommand{\href}[2]{\orighref{#1}{#2\,\smaller[4]\faExternalLink}}

\let\Red=\alert
\definecolor{few-gray-bright}{HTML}{010202}
\definecolor{few-red-bright}{HTML}{EE2E2F}
\definecolor{few-green-bright}{HTML}{008C48}
\definecolor{few-blue-bright}{HTML}{185AA9}
\definecolor{few-orange-bright}{HTML}{F47D23}
\definecolor{few-purple-bright}{HTML}{662C91}
\definecolor{few-brown-bright}{HTML}{A21D21}
\definecolor{few-pink-bright}{HTML}{B43894}

\definecolor{few-gray}{HTML}{737373}
\definecolor{few-red}{HTML}{F15A60}
\definecolor{few-green}{HTML}{7AC36A}
\definecolor{few-blue}{HTML}{5A9BD4}
\definecolor{few-orange}{HTML}{FAA75B}
\definecolor{few-purple}{HTML}{9E67AB}
\definecolor{few-brown}{HTML}{CE7058}
\definecolor{few-pink}{HTML}{D77FB4}

\definecolor{few-gray-light}{HTML}{CCCCCC}
\definecolor{few-red-light}{HTML}{F2AFAD}
\definecolor{few-green-light}{HTML}{D9E4AA}
\definecolor{few-blue-light}{HTML}{B8D2EC}
\definecolor{few-orange-light}{HTML}{F3D1B0}
\definecolor{few-purple-light}{HTML}{D5B2D4}
\definecolor{few-brown-light}{HTML}{DDB9A9}
\definecolor{few-pink-light}{HTML}{EBC0DA}

\colorlet{alert-color}{few-red-bright!80!black}
\colorlet{comment}{few-blue-bright}
\colorlet{string}{few-green-bright}

\lstdefinestyle{ccode}{
    showstringspaces=false,
    stringstyle={\ttfamily\color{string}},
    language=C,escapeinside=`',columns=flexible,commentstyle=\color{comment},
    basicstyle=\ttfamily,
    classoffset=2, keywordstyle=\color{alert-color}
}

\lstnewenvironment{ccode}[1][]%
    {\lstset{style=ccode,basicstyle=\ttfamily\openup-.17\baselineskip,#1}}%
    {}

\lstset{
  basicstyle=\itshape,
  xleftmargin=3em,
  literate={->}{$\rightarrow$}{2}
           {α}{$\alpha$}{1}
           {δ}{$\delta$}{1}
           {ε}{$\epsilon$}{1}
}

\renewcommand{\baselinestretch}{1.1}
\setlength{\parindent}{0pt}
\setlength{\parskip}{1em}

\title{INF333 2023-2024 Spring Semester}
\author{
\textbf{\color{blue}{Şirinler}} {\small(your group name, use an appropriate color)}
\\ Student 1 Name <stu1@gsu.edu.tr>
\\ Student 2 Name <stu2@gsu.edu.tr>}

\begin{document}

\maketitle

\section*{\LARGE Homework I \\
Design Document}

Please provide answers inline in a \texttt{quote} environment.


\section{Preliminaries}

\textbf{Q1:} If you have any preliminary comments on your submission, notes for
the TAs, or extra credit, please give them here.
\begin{quote}
  Answer here
\end{quote}


\textbf{Q2:} Please cite any offline or online sources you consulted while
preparing your submission, other than the Pintos documentation, course text, and
lecture notes.

\begin{quote}
  Answer here
\end{quote}


\section{Sleep}

\subsection{Data Structures}

\textbf{Q3:} Copy here the declaration of each new or changed `struct' or
`struct' member, global or static variable, `typedef', or enumeration.

Identify the purpose of each in 25 words or less.
\begin{quote}
  Answer here
\end{quote}


\subsection{Algorithms}


\textbf{Q4:} Briefly describe your implementation of \texttt{thread\_join()} and
how it interacts with thread termination.

\begin{quote}
  Answer here
\end{quote}

\textbf{Q5:} What steps are taken to minimize the amount of time spent in the
timer interrupt handler?

\begin{quote}
  Answer here
\end{quote}


\subsection{Synchronization}

\textbf{Q6:} Consider parent thread \texttt{P} with child thread \texttt{C}.
How do you ensure proper synchronization and avoid race conditions when
\texttt{P} calls \texttt{wait(C)} before \texttt{C} exits?  After \texttt{C}
exits?  How do you ensure that all resources are freed in each case?  How about
when \texttt{P} terminates without waiting, before \texttt{C} exits?  After
\texttt{C} exits?  Are there any special cases?

\begin{quote}
  Answer here
\end{quote}

\textbf{Q7:} How are race conditions avoided when multiple threads call
\texttt{timer\_sleep()} simultaneously?

\begin{quote}
  Answer here
\end{quote}

\textbf{Q8:} How are race conditions avoided when a timer interrupt occurs
during a call to \texttt{timer\_sleep()}?

\begin{quote}
  Answer here
\end{quote}


\subsection{Rationale}

\textbf{Q9:} Critique your design, pointing out advantages and disadvantages in
your design choices.

\begin{quote}
  Answer here
\end{quote}




\section{Priority Scheduling}

\subsection{Data Structures}

\textbf{Q10:} Copy here the declaration of each new or changed `struct' or
`struct' member, global or static variable, `typedef', or enumeration.  Identify
the purpose of each in 25 words or less

\begin{quote}
  Answer here
\end{quote}

\textbf{Q11:} Describe the sequence of events when a call to
\texttt{lock\_acquire()} causes a priority donation.  How is nested donation
handled?

\begin{quote}
  Answer here
\end{quote}

\textbf{Q12:} Describe the sequence of events when \texttt{lock\_release()} is
called on a lock that a higher-priority thread is waiting for.

\begin{quote}
  Answer here
\end{quote}

\subsection{Synchronization}
 
\textbf{Q13:} Describe a potential race in \texttt{thread\_set\_priority()} and
explain how your implementation avoids it. Can you use a lock to avoid this
race?

\begin{quote}
  Answer here
\end{quote}

\subsection{Rationale}

\textbf{Q14:} Why did you choose this design?  In what ways is it superior to
another design you considered?

\begin{quote}
  Answer here
\end{quote}

\section{Advanced Scheduler}

\subsection{Data Structures}

\textbf{Q15:} Copy here the declaration of each new or changed `struct' or
`struct' member, global or static variable, `typedef', or enumeration.  Identify
the purpose of each in 25 words or less.

\begin{quote}
  Answer here
\end{quote}

\subsection{Algorithms}

\textbf{Q16:} Suppose threads A, B, and C have nice values 0, 1, and 2.  Each
has a recent\_cpu value of 0.  Fill in the table below showing the scheduling
decision and the priority and recent\_cpu values for each thread after each
given number of timer ticks:

\small
\begin{Verbatim}[frame=single]
timer  recent_cpu    priority   thread
ticks   A   B   C   A   B   C   to run
-----  --  --  --  --  --  --   ------
  0
  4
  8
 12
 16
 20
 24
 28
 32
 36
\end{Verbatim}


\textbf{Q17:} Did any ambiguities in the scheduler specification make values in
the table uncertain?  If so, what rule did you use to resolve them?  Does this
match the behavior of your scheduler?

\begin{quote}
  Answer here
\end{quote}


\textbf{Q18:} How is the way you divided the cost of scheduling between code
inside and outside interrupt context likely to affect performance?

\begin{quote}
  Answer here
\end{quote}

\subsection{Rationale}

\textbf{Q19:} Briefly critique your design, pointing out advantages and
disadvantages in your design choices.  If you were to have extra time to work on
this part of the project, how might you choose to refine or improve your design?

\begin{quote}
  Answer here
\end{quote}

\textbf{Q20:} The assignment explains arithmetic for fixed-point math in detail,
but it leaves it open to you to implement it.  Why did you decide to implement
it the way you did?  If you created an abstraction layer for fixed-point math,
that is, an abstract data type and/or a set of functions or macros to manipulate
fixed-point numbers, why did you do so?  If not, why not?

\begin{quote}
  Answer here
\end{quote}

\section{Survey Questions}

Answering these questions is optional, but it will help us improve the course in future quarters.  Feel free to tell us anything you want--these questions are just to spur your thoughts.  You may also choose to respond anonymously in the course evaluations at the end of the quarter.

\textbf{Q1:} In your opinion, was this assignment, or any one of the three
problems in it, too easy or too hard?  Did it take too long or too little time?

\begin{quote}
  Answer here
\end{quote}

\textbf{Q2:} Did you find that working on a particular part of the assignment
gave you greater insight into some aspect of OS design?

\begin{quote}
  Answer here
\end{quote}

\textbf{Q3:} Is there some particular fact or hint we should give students in
future quarters to help them solve the problems?  Conversely, did you find any
of our guidance to be misleading?

\begin{quote}
  Answer here
\end{quote}

\textbf{Q4:} Do you have any suggestions for the TAs to more effectively assist
students, either for future quarters or the remaining projects?

\begin{quote}
  Answer here
\end{quote}

\textbf{Q5:} Any other comments?

\begin{quote}
  Answer here
\end{quote}


\end{document}
